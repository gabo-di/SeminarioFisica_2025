\documentclass[aspectratio=169]{beamer}
\usetheme[numbering=fraction,progressbar=frametitle]{metropolis}
\usepackage{animate}
\usepackage{graphicx}
\usepackage{physics}
\usepackage{amsmath, amssymb}
\usepackage[newfloat]{minted} % code highlighting (needs -shell-escape and activate conda env base)
\usepackage{textgreek}
\newcommand{\im}{\mathrm{i}} % for imaginary number
\usepackage[svgnames]{xcolor}
\setbeamercolor{background canvas}{bg=Ivory}
\usepackage{bm}

\title{Nonlinear Interactions and Collective Dynamics}
\author{Gabriel D\'iaz Iturry}
\date{Conferencias de F\'isica Te\'orica, FCyT 2025}



\begin{document}
\maketitle

\begin{frame}
\frametitle{Table of Contents}
\tableofcontents
\end{frame}


%%%%%%%%%%%%%%%%%%%%%%%%%%%%%%%%%%
% Chain of Nonlinear Oscillators %
%%%%%%%%%%%%%%%%%%%%%%%%%%%%%%%%%%
\section{Chain of Nonlinear Oscillators}

\begin{frame}{Chain of Nonlinear Oscillators}
    Periodic Linear Chain of N Nonlinear Oscillators Differential Equation:
    \begin{eqnarray}
        \begin{bmatrix}
            \ddot x_1 \\ \ddot x_2 \\ \ddot x_3 \\ \vdots \\ \ddot x_{N-1}  \\ \ddot x_N 
        \end{bmatrix}   & = & 
        \omega^2
        \begin{bmatrix}
            -2 & 1 & 0 & \cdots & 0 & 1\\
            1 & -2 & 1 & \cdots & 0 & 0\\  
            0 & 1 & -2 & \cdots & 0 & 0\\      
            \vdots & \vdots & \vdots & \ddots & \vdots & \vdots \\
            1 & 0 & 0 & \cdots & 1 & -2
        \end{bmatrix}
        \begin{bmatrix}
            x_1 \\  x_2 \\  x_3 \\ \vdots \\ x_{N-1} \\  x_N 
        \end{bmatrix} 
        -
        \begin{bmatrix}
            f(x_1) \\ f(x_2) \\ f(x_3) \\ \vdots \\ f(x_{N-1}) \\ f(x_N) 
        \end{bmatrix} \\
        \pause
        f(x) & = & ax^3 + bx
    \end{eqnarray}
\end{frame}

\begin{frame}{Chain N Linear Oscillators random position}
    \centering
    \animategraphics[controls,width=\linewidth]{10}{../day3/frames_lc/N_ho-}{0001}{0100}
\end{frame}

\begin{frame}{N Nonlinear Oscillators random position}
    \centering
    \animategraphics[controls,width=\linewidth]{10}{../day3/frames_no/N_ho-}{0001}{0100}
\end{frame}

\begin{frame}{Chain N Nonlinear Oscillators random position}
    \centering
    \animategraphics[controls,width=\linewidth]{10}{../day3/frames_nloc/N_ho-}{0001}{0100}
\end{frame}

%%%%%%%%%%%%%%%%%%%%%%%%%%%%%%%%%%
% Nonlinear Schrodinger Equation %
%%%%%%%%%%%%%%%%%%%%%%%%%%%%%%%%%%
\section{Nonlinear Schr\"odinger Equation}

\begin{frame}{Nonlinear Schr\"odinger Equation}
    Nonlinear Schr\"odinger Equation in position basis (Gross–Pitaevskii Equation):
    \begin{eqnarray}
        \im \hbar \frac{\partial \psi}{\partial t} & = & 
            -\frac{\hbar^2}{2m}\nabla^2 \psi + V(\bm{x}) \psi + g |\psi|^2 \psi
    \end{eqnarray}
    \pause
    \begin{itemize}
        \item $g<0$  gives focussing nonlinearity. 
        \item $g>0$  gives nonfocussing nonlinearity. 
    \end{itemize}
\end{frame}

\begin{frame}{Solitons in NSLE}
    If we consider the 1D case $V(\bm{x})=0$ then a solution of NSLE is:
    \begin{itemize}
        \item Case $g<0$, bright soliton:
        \begin{eqnarray}
            \psi(x,t) & = &
                |\psi_0| \exp\left( - \frac{\im g |\psi_0|^2 t}{2\hbar} \right)\sech\left(\frac{\sqrt{-m g} |\psi_0|}{\hbar} x\right) \nonumber
        \end{eqnarray}
        \item Case $g>0$, dark soliton:
        \begin{eqnarray}
            \psi(x,t) & = &
                |\psi_0| \exp\left( - \frac{\im g |\psi_0|^2 t}{\hbar} \right) \tanh\left( \frac{\sqrt{mg}|\psi_0|}{\hbar}  x\right) \nonumber  
        \end{eqnarray}
    \end{itemize}
\end{frame}

\begin{frame}[fragile]{N Oscillators Code}
    \begin{minted}[fontsize=\footnotesize]{julia1}
        function nlse!(du, u, p, t)
            A = p.A
            g = p.g
            r_du, i_du = eachslice(du, dims=1)
            r_u, i_u = eachslice(u, dims=1)

            # linear part 
            mul!(r_du, A, i_u)
            mul!(i_du, A, -r_u)

            # non linear part 
            map!((z,x,y)-> z + g*(x^2+y^2)*y, r_du, r_du, r_u, i_u)
            map!((z,x,y)-> z - g*(x^2+y^2)*x, i_du, i_du, r_u, i_u)
            return nothing
        end
    \end{minted}
\end{frame}

\begin{frame}{SE finite barrier}
    \centering
    \animategraphics[controls,width=\linewidth]{20}{../day3/frames_SE_finitebarrier/wave_change-}{0001}{0500}
\end{frame}

\begin{frame}{NSLE $g<0$ finite barrier}
    \centering
    \animategraphics[controls,width=\linewidth]{20}{../day3/frames_NLSE_finitebarrier/wave_change-}{0001}{0500}
\end{frame}

%%%%%%%%%%%%%%%%%%%%%%%%%%%%%%%%%%%%
% Reaction Diffusion Barkley model %
%%%%%%%%%%%%%%%%%%%%%%%%%%%%%%%%%%%%

\section{Reaction Diffusion Barkley model}

\begin{frame}{Reaction Diffusion Barkley model}
    The Barkley model of reaction diffusion
    \begin{eqnarray}
        \frac{\partial u}{\partial t} & = & \frac{1}{\epsilon} u (1-u)\left(u - \frac{v+b}{a}\right) + \nabla^2 u \\
        \frac{\partial v}{\partial t} & = & u - v + D_v \nabla^2 v
    \end{eqnarray}
    \pause
    \begin{itemize}
        \item $u$ activator, fast time scale.
        \item $v$ inhibitor, slow time scale.
    \end{itemize}
\end{frame}

\begin{frame}{Reaction Diffusion Barkley model}
    \centering
    \animategraphics[controls,width=\linewidth]{20}{../day3/frames_reactiondiffusion/wave_change-}{0001}{0200}
\end{frame}

\end{document}

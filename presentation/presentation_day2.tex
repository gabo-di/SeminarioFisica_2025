\documentclass[aspectratio=169]{beamer}
\usetheme[numbering=fraction,progressbar=frametitle]{metropolis}
\usepackage{animate}
\usepackage{graphicx}
\usepackage{physics}
\usepackage{amsmath, amssymb}
\usepackage[newfloat]{minted} % code highlighting (needs -shell-escape and activate conda env base)
\usepackage{textgreek}
\newcommand{\im}{\mathrm{i}} % for imaginary number
\usepackage[svgnames]{xcolor}
\setbeamercolor{background canvas}{bg=Ivory}
\usepackage{bm}

\title{"Normal modes" in Physics equations}
\author{Gabriel D\'iaz Iturry}
\date{Conferencias de F\'isica Te\'orica, FCyT 2025}



\begin{document}
\maketitle

\begin{frame}
\frametitle{Table of Contents}
\tableofcontents
\end{frame}



%%%%%%%%%%%%%%%%%%%%%%%
% Maxwell's Equations %
%%%%%%%%%%%%%%%%%%%%%%%
\section{Maxwell's Equations}

\begin{frame}{Maxwell's Equations}
    Maxwell's Equations:
    \begin{eqnarray}
        \bm{\nabla} \cdot \bm{E} & = &  \frac{\rho}{\varepsilon_0}  \\
        \bm{\nabla} \cdot \bm{B} & = & 0 \\
        \bm{\nabla} \times \bm{E} & = & - \frac{\partial \bm{B}}{\partial t} \\
        \bm{\nabla} \times \bm{B} & = & \mu_0 \bm{J} + \mu_0 \varepsilon_0  \frac{\partial \bm{E}}{\partial t}
    \end{eqnarray}
    \pause
    Consider no charges $\rho = 0$ and no currents $\bm{J} = \bm{0}$.
    With $c=\frac{1}{\sqrt{\mu_0\varepsilon_0}}$ this gives the vector wave equations:
    \[
    \nabla^2 \mathbf E - \frac{1}{c^2}\,\partial_t^2 \mathbf E = \mathbf 0,
    \qquad
    \nabla^2 \mathbf B - \frac{1}{c^2}\,\partial_t^2 \mathbf B = \mathbf 0.
    \]
\end{frame}

\begin{frame}{Separation of variables \& Dirichlet boundary conditions}
    Consider a parallel-plate cavity on $x\!\in\![0,L]$ and a transverse field component
    $E_y(x,t)=\varphi(t)\,\phi(x)$.

    \pause
    The 1D wave equation becomes 
    \[
    \phi''(x) + k^2 \phi(x) = 0, \qquad
    \ddot{\varphi}(t) + c^2 k^2 \varphi(t) = 0.
    \]

    \pause
    At a perfect electric conductor the tangential electric field vanishes:
    For plates at $x=0,L$ this gives
    \[
    E_y(0,t)=E_y(L,t)=0 \;\Rightarrow\; \phi(0)=\phi(L)=0 \quad\text{(Dirichlet BC).}
    \]

    \pause
    \textbf{Eigenmodes.} (Normal modes)
    $\phi_n(x)=\sin\!\left(\tfrac{n\pi x}{L}\right)$,\;
    $\varphi_n(t) = \exp(\pm \im \omega_n t)$, \;
    $k_n=\tfrac{n\pi}{L}$,\;
    $\omega_n=ck_n$,\;
    $n=1,\,2,\hdots$.
\end{frame}

\begin{frame}{Finite Differences}
    On the computer the derivative can be approximated as:
    \begin{eqnarray}
        \frac{\partial \phi}{\partial x}(x_i) & \approx & \frac{\phi(x_{i+1}) - \phi(x_{i-1})}{2\Delta x} \\
        \frac{\partial^2 \phi}{\partial x^2}(x_i) & \approx & \frac{\phi(x_{i+1}) - 2 \phi(x_{i}) + \phi(x_{i-1})}{\Delta x ^2}
    \end{eqnarray}
    where we assume to have a grid discretization ${x_0,\, x_1,\cdots,\,x_N,\,x_{N+1}}$, $\Delta x = x_i - x_{i-1}$ and the function evaluated 
    on those points $\phi(x_i) = \phi_i$.
\end{frame}

\begin{frame}{Finite Differences}
    Then considering the derivative approximation + BC we obtain:
    \begin{eqnarray}
        \frac{1}{\Delta x^2}\begin{bmatrix}
            -2 & 1 & 0 & \cdots & 0 & 0\\
            1 & -2 & 1 & \cdots & 0 & 0\\  
            0 & 1 & -2 & \cdots & 0 & 0\\      
            \vdots & \vdots & \vdots & \ddots & \vdots & \vdots \\
            0 & 0 & 0 & \cdots & 1 & -2
        \end{bmatrix}
        \begin{bmatrix}
            \phi_1 \\  \phi_2 \\  \phi_3 \\ \vdots \\ \phi_{N-1} \\  \phi_N 
        \end{bmatrix} 
        & = &
        -k^2 
        \begin{bmatrix}
            \phi_1 \\  \phi_2 \\  \phi_3 \\ \vdots \\ \phi_{N-1} \\  \phi_N 
        \end{bmatrix} 
    \end{eqnarray}
    where we use the boundary condition $\phi_0=0$ and $\phi_{N+1}=0$.
\end{frame}

\begin{frame}[fragile]{Wave Equation Eigensystem Code}
    \begin{minted}[fontsize=\footnotesize]{julia1}
        function analytical_eigenstate(i_mode)
            return x -> sin(x * pi * i_mode/ p.L)
        end

        T = Float64
        p = (L = 1,
            N = 100,
            c = 1,
            )
        A = make_EM_wave_Dirichlet_matrix(T, p)
        x = LinRange(T(0), T(p.L), p.N+2)
        F = eigen(Matrix(A), sortby=abs)
    \end{minted}
\end{frame}

\begin{frame}{Wave Equation solution}
    \begin{columns}
        \begin{column}{0.5\textwidth}
            \centering 
            \begin{figure}
            \includegraphics[width=0.9\linewidth]{../day2/EM_Dirichlet_eigenstate_001.png} \\
            \caption{First eigen mode} 
            \end{figure}
        \end{column}

        \begin{column}{0.5\textwidth}
            \centering 
            \begin{figure}
            \includegraphics[width=0.9\linewidth]{../day2/EM_Dirichlet_eigenstate_003.png} \\
            \caption{Third eigen mode} 
            \end{figure}
        \end{column}
    \end{columns}
\end{frame}

\begin{frame}{Wave Equation Eigenvalues}
    \begin{columns}
        \begin{column}{0.5\textwidth}
            \centering 
            \begin{figure}
            \includegraphics[width=0.9\linewidth]{../day2/EM_Dirichlet_eigenvalues.png} \\
            \caption{Wave Equation Eigenvalues} 
            \end{figure}
        \end{column}
    
        \pause
        \begin{column}{0.5\textwidth}
            Can we do better?
            \pause
            \begin{itemize}
                \item Higher order finite difference approx
                \pause
                \item Finite Element Method
                \pause
                \item Sine Fourier Transform
            \end{itemize}
        \end{column}

    \end{columns}
\end{frame}



%%%%%%%%%%%%%%%%%%%%%%
% Diffusion Equation %
%%%%%%%%%%%%%%%%%%%%%%
\section{Diffusion Equation}

\begin{frame}{Diffusion Equation}
    Diffusion Equation:
    \begin{eqnarray}
        \frac{\partial \rho}{\partial t} & = & \nabla \cdot \left(D \nabla \rho \right)
    \end{eqnarray}
    \begin{itemize}
        \item $\rho$ particle density.
        \item $\bm{j} = - D \nabla \rho$ particle density current.
    \end{itemize}
\end{frame}

\begin{frame}{Separation of variables \& Neumann boundary conditions}
    Consider a parallel-walls cavity on $x\!\in\![0,L]$, a probability density 
    $\rho(x,t)=\varphi(t)\,\phi(x)$, and $D$ constant. 

    \pause
    Neumann boundary condition $\nabla \rho \cdot \bm{n} (x) = \nabla \rho \cdot \bm{n} (L) = 0$

    \pause
    The 1D diffusion equation becomes 
    \[
    \phi''(x) + k^2 \phi(x) = 0, \qquad
    \dot{\varphi}(t) + D k^2 \varphi(t) = 0.
    \]

    \pause
    \textbf{Eigenmodes.} (Normal modes)
    $\phi_n(x)=\cos\!\left(\tfrac{n\pi x}{L}\right)$,\;
    $\varphi_n(t) = \exp(- \tau_n t)$, \;
    $k_n=\tfrac{n\pi}{L}$,\;
    $\tau_n=Dk_n^2$,\;
    $n=0,\,1,\hdots$.
\end{frame}

\begin{frame}{Finite Differences}
    On the computer the derivative + Neumann BC can be approximated as:
    \begin{eqnarray}
        \frac{1}{\Delta x^2}\begin{bmatrix}
            -1 & 1 & 0 & \cdots & 0 & 0\\
            1 & -2 & 1 & \cdots & 0 & 0\\  
            0 & 1 & -2 & \cdots & 0 & 0\\      
            \vdots & \vdots & \vdots & \ddots & \vdots & \vdots \\
            0 & 0 & 0 & \cdots & 1 & -1
        \end{bmatrix}
        \begin{bmatrix}
            \phi_1 \\  \phi_2 \\  \phi_3 \\ \vdots \\ \phi_{N-1} \\  \phi_N 
        \end{bmatrix} 
        & = &
        -k^2 
        \begin{bmatrix}
            \phi_1 \\  \phi_2 \\  \phi_3 \\ \vdots \\ \phi_{N-1} \\  \phi_N 
        \end{bmatrix} 
    \end{eqnarray}
    where we use the boundary condition $\nabla \phi_0=0$ and $\nabla \phi_{N+1}=0$.
\end{frame}

\begin{frame}[fragile]{Difussion Equation Eigensystem Code}
    \begin{minted}[fontsize=\footnotesize]{julia1}
        function analytical_eigenstate(i_mode)
            return x -> cos(x * pi * i_mode/ p.L)
        end

        T = Float64
        p = (L = 1,
            N = 100,
            c = 1,
            )
        A = make_Diffusion_eq_Neumann_matrix(T, p)
        x = LinRange(T(0), T(p.L), p.N+2)
        F = eigen(Matrix(A), sortby=abs)
    \end{minted}
\end{frame}

\begin{frame}{Diffusion Equation solution}
    \begin{columns}
        \begin{column}{0.5\textwidth}
            \centering 
            \begin{figure}
            \includegraphics[width=0.9\linewidth]{../day2/Diffusion_eq_eigenstate_001.png} \\
            \caption{First eigen mode} 
            \end{figure}
        \end{column}

        \begin{column}{0.5\textwidth}
            \centering 
            \begin{figure}
            \includegraphics[width=0.9\linewidth]{../day2/Diffusion_eq_eigenstate_003.png} \\
            \caption{Third eigen mode} 
            \end{figure}
        \end{column}
    \end{columns}
\end{frame}



%%%%%%%%%%%%%%%%%%%%%%%%
% Schrodinger Equation %
%%%%%%%%%%%%%%%%%%%%%%%%
\section{Schr\"odinger Equation}

\begin{frame}{Schr\"odinger Equation}
    Schr\"odinger Equation in position basis:
    \begin{eqnarray}
        \im \hbar \frac{\partial \psi}{\partial t} & = & 
            -\frac{\hbar^2}{2m}\nabla^2 \psi + V(\bm{x}) \psi
    \end{eqnarray}
    \pause
    \begin{itemize}
        \item $\psi$  gives probability amplitude. 
        \item $-\frac{\hbar^2}{2m}\nabla^2 \psi$  comes from momentum contribution to the Hamiltonian. 
        \item $V(\bm{x})$ is the potential energy contribution to the Hamiltonian.
    \end{itemize}
    \pause
    We will consider $V(\bm{x}) = V_0$ for $x \in [-L, L]$ and $V(\bm{x}) = 0$ elsewhere. 
\end{frame}

\begin{frame}{Finite well $V_0 < 0$}
    Bound states
    \begin{columns}
        \begin{column}{0.5\textwidth}
            \centering 
            \begin{figure}
            \includegraphics[width=0.9\linewidth]{../day2/Schr_finitewell_eigenstate_001.png} \\
            \caption{First bound state} 
            \end{figure}
        \end{column}

        \begin{column}{0.5\textwidth}
            \centering 
            \begin{figure}
            \includegraphics[width=0.9\linewidth]{../day2/Schr_finitewell_eigenstate_002.png} \\
            \caption{Second bound state} 
            \end{figure}
        \end{column}
    \end{columns}
\end{frame}

\begin{frame}{Finite well $V_0 < 0$}
    No bound states
    \begin{columns}
        \begin{column}{0.5\textwidth}
            \centering 
            \begin{figure}
            \includegraphics[width=0.9\linewidth]{../day2/Schr_finitewell_eigenstate_003.png} \\
            \caption{First no bound state} 
            \end{figure}
        \end{column}

        \begin{column}{0.5\textwidth}
            \centering 
            \begin{figure}
            \includegraphics[width=0.9\linewidth]{../day2/Schr_finitewell_eigenstate_004.png} \\
            \caption{Second no bound state} 
            \end{figure}
        \end{column}
    \end{columns}
\end{frame}

\begin{frame}{Finite barrier $V_0 > 0$}
    Near degenerated eigen states
    \begin{columns}
        \begin{column}{0.5\textwidth}
            \centering 
            \begin{figure}
            \includegraphics[width=0.9\linewidth]{../day2/Schr_finitebarrier_eigenstate_001.png} \\
            \caption{First eigen state} 
            \end{figure}
        \end{column}

        \begin{column}{0.5\textwidth}
            \centering 
            \begin{figure}
            \includegraphics[width=0.9\linewidth]{../day2/Schr_finitebarrier_eigenstate_002.png} \\
            \caption{Second eigen state} 
            \end{figure}
        \end{column}
    \end{columns}
\end{frame}

\begin{frame}{Sum of near degenerated eigen states}
    \centering
    \animategraphics[controls,width=\linewidth]{10}{../day2/frames_Schr_finitebarrier/sum_modes_01-}{0001}{0100}
\end{frame}



\end{document}
